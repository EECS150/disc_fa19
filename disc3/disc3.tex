\documentclass[12pt,aspectratio=169]{beamer}

\usetheme{metropolis}
\setbeamertemplate{bibliography item}{\insertbiblabel}
\usepackage{xcolor}
\usepackage{appendixnumberbeamer}
\usepackage{verbatim}
\usepackage{caption}
\usepackage{minted}
\setsansfont[BoldFont={Fira Sans Bold}, ItalicFont={Fira Sans Italic}]{Fira Sans Regular}
\usepackage[backend=biber,style=ieee]{biblatex}
\usepackage{hyperref}
\usepackage{circuitikz}
\ctikzset{tripoles/pmos style/emptycircle}
\ctikzset{tripoles/mos style/arrows}
\graphicspath{{./figs/}}
\setlength{\belowcaptionskip}{-10pt}
\setbeamersize{text margin left=5mm,text margin right=7mm}

\title{EECS 151/251A: Discussion 3}
\subtitle{Verilog Simulation, Parameterized Modules}
\author{}
\date{9/12/2019}

\begin{document}
\begin{frame}
    \maketitle
\end{frame}

\begin{frame}[fragile]{Simulation Basics}
Let's test a very simple Verilog module.
  \begin{columns}
    \begin{column}{0.5\textwidth}
      \begin{center}
      \begin{minted}{verilog}
    module adder(
        input [31:0] a,
        input [31:0] b,
        output [31:0] c
    );
        assign c = a + b;
    endmodule
      \end{minted}
      \end{center}
    \end{column}
    \begin{column}{0.5\textwidth}
      \begin{itemize}
        \item The \verb|adder| module is \textit{synthesizable}; it belongs on an ASIC or FPGA
        \item In a testbench, we refer to the \verb|adder| module as the \textit{DUT} (device under test)
        \item Note Verilog's truncation rules: \verb|a + b| produces a 33 bit number, which is truncated to 32 bits when assigned to \verb|c|
      \end{itemize}
    \end{column}
  \end{columns}
\end{frame}

\begin{frame}[fragile]{The Testbench}
  \begin{center}
    \begin{minted}[fontsize=\small]{verilog}
module adder_tester();
  reg [31:0] a, b;
  wire [31:0] c;
  adder dut (.a(a), .b(b), .c(c)); // device under test
  // ... initial block of adder_tester
endmodule
  \end{minted}
  \begin{itemize}
    \item The \verb|adder_tester| module is \textbf{NOT} \textit{synthesizable}; it is executed by an \textit{RTL simulator}
    \item Note we have created \verb|reg| nets to drive the DUT's inputs and \verb|wire| nets to sense the DUT's outputs
  \end{itemize}
  \end{center}
\end{frame}

\begin{frame}[fragile]{Testbench Initial Block}
  \begin{columns}
    \begin{column}{0.5\textwidth}
      \begin{center}
        \begin{minted}[fontsize=\footnotesize]{verilog}
initial begin
    a = 32'd1;
    b = 32'd2;
    #1;
    if (c != 32'd3) begin
        $display("FAILED");
    end
    a = 32'd5;
    b = 32'd10;
    #1;
    $finish();
end
      \end{minted}
      \end{center}
    \end{column}
    \begin{column}{0.5\textwidth}
      \begin{itemize}
        \item The \verb|initial| block defines the `entry point' of the simulator
        \item The code in the \verb|initial| block is run \textit{sequentially}, just like software
        \item The DUT is driven using blocking (=) assignments
        \item \verb|#(n);| is used to advance simulation time by \verb|n| timesteps
        \item \verb|$display| and \verb|$finish| are system functions
      \end{itemize}
    \end{column}
  \end{columns}
\end{frame}

\begin{frame}[fragile]{Timescales and Timesteps}
  \begin{center}
    \begin{minted}[fontsize=\small]{verilog}
`timescale 1ns/10ps
`timescale (simulation time step)/(simulation time resolution)
    \end{minted}
    \begin{itemize}
      \item At the top of each testbench, there's a \verb|`timescale| declaration
      \item The simulation time step defines how much time is advanced when running \verb|#1;|
      \item The simulation time resolution defines the smallest amount time can be advanced
      \item In the example, \verb|#1;| = 1ns and \verb|#0.010;| is the smallest delay possible
    \end{itemize}
  \end{center}
\end{frame}

\begin{frame}[fragile]{Adder Testbench Demo}
  \begin{block}{Demo}
    Run through the adder testbench. See the waveform and play with time delays.
  \end{block}
\end{frame}

\begin{frame}[fragile]{4-State Signals in Verilog}
    \begin{minted}[fontsize=\footnotesize]{verilog}
module counter(input clk);
    reg [3:0] count;
    always @(posedge clk) begin
        count <= count + 'd1;
    end
endmodule
    \end{minted}
  \begin{itemize}
    \item All \textit{registers} begin with an initial value of `x' in simulation unless otherwise specified
    \item Every signal in Verilog has 4 potential states: 0, 1, `x` (unknown), `z' (high-impedance/unconnected)
    \item We can set initial values of \textit{registers} with \verb|initial count = 0|
    \item Initial values are synthesizable on some FPGAs but not on ASICs, so using a \verb|reset| is recommended
  \end{itemize}
\end{frame}

\begin{frame}[fragile]{Simulation Constructs}
  Wait for 2 rising edges of \verb|signal|:
  \begin{center}
    \begin{minted}[fontsize=\footnotesize]{verilog}
@(posedge signal);
@(posedge signal);
    \end{minted}
  \end{center}

  Wait for 10 rising \verb|clk| edges:
  \begin{center}
    \begin{minted}[fontsize=\footnotesize]{verilog}
repeat (10) @(posedge clk);
    \end{minted}
  \end{center}

  Generate a clock signal:
  \begin{minted}[fontsize=\footnotesize]{verilog}
reg clk;
initial clk = 0;
always #(10) clk <= ~clk;
  \end{minted}
\end{frame}

\begin{frame}[fragile]{\$display}
  The \verb|$display()| system function is similar to \verb|printf()| in C.
    \begin{minted}[fontsize=\footnotesize]{verilog}
$display("Wire x in decimal is %d", x);
$display("Wire x in binary is %b", x);
    \end{minted}

  \verb|$display()| can be used in \verb|initial| blocks as well as inside RTL.
  \begin{minted}[fontsize=\footnotesize]{verilog}
module x(input clk, input valid, input data);
  always @(posedge clk) begin
    if (valid) $display("Data is %d", data);
  end
endmodule
  \end{minted}
\end{frame}

\begin{frame}[fragile]{Counter Demo}
  \begin{block}{Demo}
    Run through the counter testbench. Deal with unknown values, initialize registers, add a reset. Use \verb|$display| in the DUT.
  \end{block}
\end{frame}

\begin{frame}[fragile]{Generate Macros}
  Generate macros (\verb|generate for| and \verb|generate if|) can be used to programmatically instantiate hardware.
  Stamp out \verb|n| instances of \verb|mod| and connected them to wires \verb|k| and \verb|s|.
  \begin{minted}[fontsize=\footnotesize]{verilog}
genvar i;
generate for (i = 0; i < n; i = i + 1) begin
  mod name(.a(s[i]), .b(k[i+1]));
end endgenerate
  \end{minted}

  If there's a module parameter \verb|fast_mult| to choose multiplication speed
  \begin{minted}[fontsize=\footnotesize]{verilog}
generate if (fast_mult == 1)
  assign c = a * b;
else
  shift_and_add_mult(.a(a), .b(b), .c(c));
endgenerate
  \end{minted}
\end{frame}

\begin{frame}[fragile]{Shift Register Example}
  \begin{block}{Demo}
    Run through the shift register testbench. Work through the generate-for statement.
  \end{block}
\end{frame}

\begin{frame}[fragile]{Parameterized Modules}
  Modules can have integer parameters to make them `generators':
  \begin{minted}[fontsize=\footnotesize]{verilog}
module adder #(parameter width=32)
  (input [width-1:0] a,
   input [width-1:0] b,
   output [width:0] s);
   assign s = a + b;
endmodule
  \end{minted}

  This adder can be re-used with different parameterizations:
  \begin{minted}[fontsize=\footnotesize]{verilog}
wire [31:0] a32, b32; wire [32:0] c1;
wire [63:0] a64, b64; wire [64:0] c2;
adder #(.width(32)) adder1 (.a(a32), .b(b32), .c(c1));
adder #(.width(64)) adder2 (.a(a64), .b(b64), .c(c2));
  \end{minted}
\end{frame}

\begin{frame}[fragile]{Shift Register Example}
  \begin{block}{Demo}
    Make the shift register a parameterized module.
  \end{block}
\end{frame}

\begin{frame}{Exhaustive Testing}
  \begin{block}{Demo}
    A small adder can be tested exhaustively using nested for loops.
  \end{block}
\end{frame}

\begin{frame}{Randomized Testing}
  \begin{block}{Demo}
    A large adder can't be tested exhaustively since it would take too much time. Instead test it using random stimulus.
  \end{block}
\end{frame}

\begin{frame}[fragile]{2D Regs + Memories}
  \begin{minted}[fontsize=\footnotesize]{verilog}
    // A memory structure that has eight 32-bit elements
    reg [31:0] fifo_ram [7:0];
    fifo_ram[2] // The 3rd 32-bit element
    fifo_ram[5][7:0] // The lowest byte of the 6th 32-bit element
  \end{minted}

  \begin{block}{Demo}
    Test the simple memory!
  \end{block}
\end{frame}

\end{document}
